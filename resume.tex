\documentclass{res}

\usepackage{pifont}
\newsectionwidth{0pt}
\usepackage{fancyhdr}
\renewcommand{\headrulewidth}{0pt}
\setlength{\headheight}{24pt}
\setlength{\headsep}{24pt}
\pagestyle{fancy}
\cfoot{}
\topmargin=-0.3in
\usepackage[margin=0.5in]{geometry}

\begin{document}
	\thispagestyle{empty} % no header
	\centerline{\bf \Large{Nicholas VanCise}}
	\centerline{(702) 601-4856}
	\centerline{vancise@unlv.nevada.edu}
  \centerline{nicholas-vancise.glitch.me}
  \centerline{github.com/thenick775}
  \vspace{-10pt}

	\begin{resume}

		\section{{PROFESSIONAL EXPERIENCE}}
    \noindent\rule[0.5ex]{\linewidth}{1pt}
		{\bf Terbine - The Global Exchange for IoT Data} \hfill \emph{Internship, November 2018 - Present} \\
			\emph{Data Linker}

			\begin{itemize} \itemsep -2pt
				\item Develops and links real time data feeds into the continuous ingestion system
				\item Develops and maintains back-end infrastructure for scalable distributed data ingestion and processing
				\item Develops data visualizations based on archival and continuous data feeds
				\item Enables handling of exotic file types, file manipulation, and large static data files
			\end{itemize} \vspace{-2mm}

			\emph{Data Searcher}

			\begin{itemize} \itemsep -2pt
				\item Responsible for exploration of new IoT data sources and feeds to be ingested into the system
				\item Ensures individual feeds adhere to the Metadata Specifications, and that all ancillary information is reviewed
			\end{itemize} \vspace{-2mm}

		{\bf Academic Success Center, UNLV} \hfill \emph{Summers of 2017, 2018, 2019} \\
			\emph{Team Lead}

			\begin{itemize} \itemsep -2pt
				\item Managed planning and presentation of lectures, bookkeeping, and dynamic of the ALEKS program
				\item Developed individual lesson plans based on statistical assessment of student performance
        \item Prepared and proctored ALEKS placement exam
			\end{itemize}

\vspace{-3pt}

		\section{{INDUSTRY PROJECTS}}
    \noindent\rule[0.5ex]{\linewidth}{1pt}
		{\bf Ingestion API} \hfill \vspace{5pt} \\
			The Ingestion API is designed to function as the middle man between independent programs orchestrated by Apache Airflow that collect data, and multiple postgres database connections. This API was built with golang.
\vspace{-5pt}

		{\bf Airflow Ingestion Cluster} \hfill \vspace{5pt} \\
			The Airflow Ingestion Cluster is designed to fetch and process data from a multitude of public data sources using docker based web scrapers, where this data is sent to the Ingestion API for storage. This cluster was built with Apache Airflow and Docker, where the web scrapers were built using python, golang, and bash.
\vspace{-5pt}

		{\bf Ingestion Index Crawler} \hfill \vspace{5pt}\\
			This crawler is designed to reduce search times of already ingested data. It crawls all instances for a specified user, and produces a list that can be easily and quickly searched. This software was built using Docker and Selenium in python.
\vspace{-5pt}

		{\bf IoT Data Visualizations} \\
    \emph{github.com/thenick775/terbine\_visualizations} \hfill \vspace{4pt}\\
			These projects are an interactive way to view and analyze data from a multitude of physical sensors located around the world. The indivudual visualizations were constructed using Javascript, Jupyter-Notebooks, D3, Kepler.gl, and Leaflet, where they are currently featured on the main Terbine website.

\vspace{-3pt}

		\section{{PUBLIC PROJECTS}}
    \noindent\rule[0.5ex]{\linewidth}{1pt}
		{\bf Metroidvania} \\
			\emph{github.com/thenick775/metroidvaniafangame} \hfill \vspace{3pt}\\
			This project is a small game written in Objective-C that utilizes features from Spritekit, GameplayKit, AVAudioPlayer, and JSTileMap. I have written all of the event driven animation scheme, collision detection, character physics, data storage schemes, and game logic.
\vspace{-4pt}

		{\bf Terbine Map Visualization} \\
			\emph{github.com/thenick775/Terbine-Map} \hfill \vspace{3pt} \\
			This was a fun exercise in data visualization, where fixed coordinate data points in Terbine were plotted and connected on an interactive world map. The data mining was done using Selenium in python, and data visualization was accomplished using Mapbox in R.

		\section{{EDUCATION}}
    \noindent\rule[0.5ex]{\linewidth}{1pt}
		{\sl Bachelor of Science}, Computer Science \hfill \emph{Class of Dec 2020} \\
		University of Nevada Las Vegas \hfill \emph{GPA: 3.774} \\ \vspace{-4mm}

		\section{{RELEVANT SKILLS}}
    \noindent\rule[0.5ex]{\linewidth}{1pt}
			{\bf Languages:} Golang, Python, Bash, Objective-C, C, Matlab, R, Javascript \\
			{\bf Related Technologies:} Docker, Amazon Web Services (AWS), Apache Airflow, Selenium, D3, Git

	\end{resume}
\end{document}




